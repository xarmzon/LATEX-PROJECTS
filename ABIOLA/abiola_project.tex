\documentclass[11pt]{report}
\usepackage{amsmath}
\usepackage{amssymb}
\usepackage{bbm}
\usepackage{graphicx}
\usepackage{tikz}
\usepackage{enumitem}
\usepackage{graphicx}
\usepackage{longtable}
\usepackage{array}

\newcommand{\ubt}[1]{\textbf{\underline{#1}}}
\newcommand{\sps}{\\[0.2cm]}
\newcommand{\spn}[1]{\\[#1cm]}
\newcommand{\refn}[1]{(\ref{#1})}
\newcommand{\refx}[1]{\refn{eq:#1}}
\newcommand{\bt}[1]{\textbf{#1}}
\newcommand{\dsp}{\displaystyle}
\newcommand{\NI}{\noindent}
%\newcommand{\real}{ \mathbb{R}}
\newcommand{\sprime}{'}
\newcommand{\dprime}{''}
\newcommand{\tprime}{'''}
\newcommand{\mbf}[1]{\mathbf{#1}}
\newcommand{\sbracket}[1]{\left[#1\right]}
\newcommand{\example}[1]{\section*{\ubt{Example #1}}}
\newcommand{\property}{\subsubsection{\ubt{Property}}}
\newcommand{\properties}{\subsubsection{\ubt{Properties}}}
\newcommand{\solution}{\subsubsection{\ubt{Solution}}}
\newcommand{\proposition}[1]{\section*{\ubt{Proposition #1}}}
\newcommand{\eg}{\section*{\ubt{Example}}}

\newcommand{\real}{\mathbbm{R}}

\renewcommand{\baselinestretch}{1.5}
\renewcommand{\contentsname}{Table of Contents}


%\setlength{\parindent}{1em}


\begin{document}
	
	%%%%%%%%%%%%%%%%%%%FRONT COVER%%%%%%%%%%%%%%%%%%%
	\addcontentsline{toc}{chapter}{TITLE PAGE}
	\clearpage
	\thispagestyle{empty}
	\begin{center}
		\Large \bt{SOLUTION OF SECOND ORDER ORDINARY DIFFERENTIAL EQUATION USING ADOMIAN DECOMPOSITION}
	\end{center}

	\hspace{7cm}
	
	\begin{center}
		\textbf{\textit{BY}}
	\end{center}
	
	\hspace{5cm}
	
	\begin{center}
		\large \textbf{ELUTADE, ISAIAH ABIOLA
			\\
			17/56EB050}
	\end{center}
	
	\hspace{9cm}
	
	\begin{center}
		A PROJECT SUBMITTED TO THE DEPARTMENT OF MATHEMATICS, FACULTY OF PHYSICAL SCIENCES, UNIVERSITY OF ILORIN, ILORIN, KWARA STATE, NIGERIA.
	\end{center}

	\hspace{7cm}
	
	\begin{center}
		IN PARTIAL FULFILMENT OF REQUIREMENTS FOR THE AWARD OF BACHELOR OF SCIENCE (B. Sc.) DEGREE IN MATHEMATICS.
	\end{center}
	\hspace{5cm}
	\\ \\ 
	\begin{center}
		\textbf{February, 2022}
	\end{center}

	\newpage
	\pagenumbering{roman}
	\addcontentsline{toc}{chapter}{CERTIFICATION}
	\section*{\begin{center}\textbf{\Large CERTIFICATION}   \end{center}}
	This is to certify that this project was carried out by \textbf{ELUTADE, Isaiah Abiola} with Matriculation Number  17/56EB050 in the Department of Mathematics, Faculty of Physical Sciences, University of Ilorin, Ilorin, Nigeria, for the award of Bachelor of Science (B.Sc.) degree in Mathematics.
	\\
	\\
	................................... \qquad \qquad\qquad\qquad\qquad\qquad...................... \\
	Prof. A. O. Taiwo   \quad\qquad\qquad\qquad\qquad\qquad\qquad\qquad Date\\
	Supervisor\\
	\\
	\\
	\\
	...................................... \qquad\qquad\qquad\qquad\qquad\qquad ......................\\
	Prof. K. Rauf      \qquad\qquad\qquad\qquad\qquad\qquad\qquad\qquad\quad     Date\\
	Head of Department\\
	\\
	\\
	\\
	..................................... \qquad\qquad\qquad\qquad\qquad\qquad .......................\\
	Prof.o \quad\qquad\qquad\qquad\qquad\qquad\qquad\qquad         Date\\
	External Examiner 
	
	\newpage
	%%ACKNOLEDGMENTS%%
	\section*{\begin{center}\textbf{\Large ACKNOWLEDGMENTS}\end{center}}
	\addcontentsline{toc}{chapter}{ACKNOWLEDGMENTS}
	All glory, honour and adoration to Almighty God for His love, grace, mercies, and provisions throughout my stay in school. Thus far He hath kept me.\\
	
	\NI My sincere appreciation to my supervisor, Prof O. A. Taiwo, for his fatherly care and support toward the completion of this project and for his highly exemplary leadership. I want to acknowledge my Head of Department, Prof K. A. Raufa and my level adviser, Dr. K. A Bello for their love and guidance.
	
	\NI I will like to appreciate the efforts of my lecturers across all levels: Prof. J. A. Gbadeyan, Prof. T. O. Opoola, Prof. O. M. Bamigbola, Prof. M. O. Ibrahim, Prof. R. B. Adeniyi, Prof. M. S. Dada, Prof. K. O. Babalola, Prof. A. S. Idowu, Prof. (Mrs) O. A. Fadipe-Joseph, Dr. E. O. Titiloye, Dr. (Mrs) Y. O. Aderinto, Dr (Mrs) C. N. Ejieji, Dr. B. M. Yisa, Dr. J. U. Abubakar, Dr. (Mrs) G. N. Bakare, Dr. (Mrs) I. F. Usamot, Dr. B. M. Ahmed, Dr. O. T. Olootu, Dr. O. A. Uwaheren, Dr. T. L. Oyekunle, Dr. O. Odetunde, Dr. A. Y. Ayinla. The efforts of all the non-academic staff of the department are equally appreciated.\\
	
	\NI I, with all love and sincerity, want to appreciate the love and great sacrifice of my parents, Deacon E. O. Elutade and Mrs. D. O. Elutade. You're indeed a blessing to me.\\
	
	\NI I appreciate my siblings, Elutade Solomon and Elutade Oluwatoyin for their full support and prayers through the journey. I sincerely appreciate the love and care of my personal person Omotosho Kikelomo. Thanks for believing in me.\\
	
	\NI To my friends, Jimoh, Bolemm, Visticks, Troy, Precious, Deborah and Sir White, thanks for all you do.\\
	
	\NI I am grateful for Elder P. O. Adeyemi for his love and care.\\
	
	\NI My special regards to President Patrick and Tutor Rasta and Tosin. God bless you real big.\\
	
	\NI TACSFON Unilorin chapter, the clique, ARC, Alpha Class, Mathematics Students Christian Fellowship (MSCF) family, thanks so much for the love and prayers. May God Almighty bless you all. Amen.
	
	\newpage
	%%DEDICATION%%
	\section*{\begin{center}\textbf{\Large DEDICATION}\end{center}}
	\addcontentsline{toc}{chapter}{DEDICATION}
	The project work is dedicated to Almighty God, my parent and my siblings.
	
	\newpage
	%%ABSTRACT%%
	\section*{\begin{center}\textbf{\Large ABSTRACT}\end{center}}
	\addcontentsline{toc}{chapter}{ABSTRACT}
	In this project, we use Adomian decomposition method to find solution to second order ordinary differential equation. With the examples solved and the results obtained, it shows that Adomian decomposition method converges to the exact solution of the problem.
	
	\newpage
	%%%%%%%%%%%%%%%%%%%TABLE OF CONTENTS%%%%%%%%%%%%%%%%%%%
	\addcontentsline{toc}{chapter}{TABLE OF CONTENTS}
	\tableofcontents
	
	\newpage
	\pagenumbering{arabic}
	%%%%%%%%%%%%%%%%%%%CHAPTER ONE%%%%%%%%%%%%%%%%%%%
	\chapter{Introduction to Adomian Decomposition Method (ADM)}	
	\section{General Introduction}
	The Adomian Decomposition Method (ADM) was first established in the 1980's by George Adomian, chairman of Center for Applied Mathematics at the University of Georgia. In the recent years, this method of Adomian Decomposition has be paid attention to in the field of applied mathematics and series solution. In addition, the ADM is widely used to obtain the solution to many types of linear or non-linear ordinary differential equation and integral equations.
	
	The ADM gives the accurate and efficient solution to problem in a direct and simply way without the use of linearization and partubation which can change the physical behaviour of the method.
	
	For the purpose of this study, we consider the solution of second order differential equation of the form 
	\begin{eqnarray*}
		P(x)\frac{d^2y}{dx^2}+ Q(x)\frac{dy}{dx} + R(x)y = F(x)
	\end{eqnarray*}
	or
	\begin{eqnarray*}
		y\dprime P(x) + y\sprime Q(x) + R(x)y=F(x)
	\end{eqnarray*}
	with initial conditions
	\begin{eqnarray*}
		y(x_0) = y_0 \text{ and } y\sprime(x_0) = y_1
	\end{eqnarray*}
	Where $P(x), Q(x), R(x)$ and $F(x)$ are continuous functions of $x$; and $y_0$ and $y_1$ are given constant.
	
	\section{Definition of Relative Terms}
	\subsection{Differential Equation}
	Differential equation is an equation that contains at least one derivative of an unknown function, either ordinary derivative $\dsp\left(\frac{d}{dy}\right)$, or a partial derivative $\dsp\left(\frac{\partial}{\partial x}\right)$.
	
	\subsection{General Second Order Ordinary Differential Equation}
	The general Second Order Ordinary Differential Equation of an independent variable $x$ and dependent variable $y$ is given by 
	\begin{eqnarray*}
		\begin{split}
			P(x)\frac{d^2y}{dx^2} + Q(x)\frac{dy}{dx} + R(x)y &= f(x) \hspace{1cm} \text{Or }\sps
			y\dprime P(x) + y\sprime Q(x) + R(x)y &= f(x)
		\end{split}
	\end{eqnarray*}
	where $P(x), Q(x), R(x),f(x)$ are continuous functions of $x$.
	
	\subsection{Initial Value Problem}
	An initial-value problem for the second order equation
	\begin{eqnarray*}
		y\dprime P(x) + y\sprime Q(x) + R(x)y = f(x)
	\end{eqnarray*}
	consists of finding a solution $y$ of the differential equation that also satisfies initial condition of the form
	\begin{eqnarray*}
		y(x_0) = y_0 \hspace{2cm} y\sprime(x_0) = y_1
	\end{eqnarray*}
	where $y_0$ and $y_1$ are given constants and $P(x), Q(x), R(x),f(x)$ are continuous functions of $x$ and $P(x)\neq 0$.
	
	\subsection{Operator}
	An operator is a function that takes a function as an argument instead of number. Examples of operators are:
	\begin{eqnarray*}
		L = \frac{d}{dx}\; ; \; L=\frac{\partial}{\partial x}\; ; \ \; L = \int dx\; ; \; L= \int_a^b dx
	\end{eqnarray*}
	
	\subsection{Adomian Polynomials $\mathbf{(A_n)}$}
	The Adomian Polynomials of a non-linear differential equation is obtained using the formula
	\begin{equation}
		A_n = \frac{1}{n!}~\frac{d^n}{d\lambda^2}\left[ N\left(\sum_{n=0}^{\infty}\lambda^n U_n\right)\right]_{\lambda=0}\hspace{1cm} n=0,1,2,\ldots
	\end{equation}

	\section{Aims and Objectives}
	\subsection{Aim}
	The aim of this project work is to adopt the method of Adomian decomposition to solve Linear and Non-linear Second Order Differential Equation of the form 
	\begin{equation*}
		y\dprime P(x) + y\sprime Q(x) + R(x)y = f(x)
	\end{equation*}
	where $P(x), Q(x), R(x),f(x)$ are continuous functions of $x$.
	
	\subsection{Objectives}
	The objectives of this study are to 
	\begin{enumerate}
		\item Describe Adomian decomposition method
		\item Use ADM to solve linear ordinary differential equation
		\item determine the solution of second order ordinary differential equation by Adomian decomposition method
	\end{enumerate}
	
	\section{Significance of the Project}
	The importance or significance of this project is to apply Adomian Decomposition Method to solve linear and non-linear ordinary differential equation of second order will converges the problem to its exact solution.
	
	\section{Outline of the Project}
	The project is divided into five chapters. Chapter one consists of the introduction, definition of relevant terms, aims and objectives, significance of the project and project outline.\\
	
	\NI Chapter two consist of literature review, Chapter three consists of methodology; method of solution of ADM; Numerical examples on Adomian Polynomial; Numerical examples of linear ordinary differential equation.\\
	
	\NI Chapter four consist of methodology, Numerical examples on Non-linear ordinary differential equation. Chapter five consist of discussion of result; conclusion and recommendation for further study.
	

	
	%%%%%%%%%%%%%%%%%%%CHAPTER TWO%%%%%%%%%%%%%%%%%%%
	\chapter{Literature Review}
	\section{Introduction}
	The study of differential equation began in 1675 when Leibniz wrote the equation
	\begin{eqnarray}
		\int x dx = \frac{1}{2}x^2
	\end{eqnarray}
	Leibniz inaugurated the differential sign $\dsp\left(\frac{dy}{dx}\right)$ and integral sign $\dsp\Big(\int\Big)$ in (1675) a hundred years before the period of initial discovery of general method of integrating ordinary differential equation ended. (Ince 1956).\\
	
	\NI According to Sasser (2005) in Orapine, Tiza and Amon(2016) the search for general methods of integrating began when Newton classified the first order differential equation into three classes:
	\begin{gather}
		\frac{dy}{dx} = f(x)\sps
		\frac{dy}{dx} = f(x,y)\sps
		x\frac{\partial u}{\partial y}+ y\frac{\partial u}{\partial y} = u
	\end{gather}
	
	\NI The first two classes contain only ordinary derivative of one or more dependent variable, with respect to a single independent variable, and are known today as ordinary differential equation. The third class involves partial derivatives of one dependent variable and today its called partial differential equation.\\
	
	\NI In the 20$^{th}$ century, Ordinary differential equation(ODE) is widely applied in many field and the numerical solution has made a great development.\\
	
	\NI Many of the problems in the field of engineering is expressed in terms of boundary value problem (BVP) which are boundary differential equation with boundary conditions and also in terms of initial value problem (IVP) which are ODE with initial condition of the unknown function at a given domain of solution. Hilderbrand (1974).\\
	
	\NI Billingham and King (2003) studied mathematical modelling and state the importance of ODE in modelling dynamics system. Saying it gives the conceptual skills to formulate, develop, solve, evaluate and validate such system. Many physical, chemical and biological system can be described using mathematical model. Once the model is formulated, we need differential equation in order to predict and quantify the features of the system modelled. Tiza (2016).\\
	
	\NI Numerical techniques are used to solved mathematical models in engineering, many branch of physics, human physiology, applied mathematics etc. Some of the mathematical techniques or method used in solving modelling problems are Cubic Spline Method, Finite Difference Methods, Taylor's Series Method, Runge-Kuta Method, Shooting Method, Pertubation method, Adomian Decomposition Method, Euler Method etch. The main method to be considered is Adomian Decomposition method, but we shall briefly discuss some of the method listed above i.e Euler Method, Runge-Kutta Method and Taylor's Series Method.
	
	\section{Runge-Kutta Method}
	This method was devise by two German Mathematician Runge and Kutta around 1900.\\
	The method is used to find numerical solution to a first order differential equation, the method uses four value of $k$ at four point in one step. Consider the equation
	\begin{eqnarray}
		y\sprime = f(x,y)
	\end{eqnarray}
	and $y(x_0)=y_0$\\
	The function $f$ and $(x_0,y_0)$ are given, now choose a step size of $h>0$ and define
	
	\begin{gather}
		y_{n+1} = y_n+\frac{h}{6}(k_1+2k_2+2k_3+k_4)\sps
		x_{n+1} = x_{n+h}\sps
		\forall n = 1,2,3,\ldots
	\end{gather}
	using 
	\begin{eqnarray}
		k_1 &=& f(x_n,y_n) \sps
		k_2 &=& f(x_n + \frac{h}{2}, y_n + \frac{h}{2}k_1)\sps
		k_3 &=& f(x_n+\frac{h}{2}, y_n \frac{h}{2}k_2)\sps
		k4 &=& f(x_n + h, y+hk_3) 
	\end{eqnarray}
	Each $k_i, i=1,2,3,\ldots$ represent $k^{th}$ order Runge-Kutta method, the forth order is the most stable and easy to implement.
	
	\section{Taylor's Series Method}
	Considering the one-dimensional initial value problem $y\sprime(x) = f(x,y)$ and $y(x_0)=y_0$ where $f$ is a function of two variables $x$ and $y$ and $(x_0,y_0)$ is a known point on the solution curve(initial points). If the existence of all higher order partial derivatives is assumed for $y$ at $x=x_0$ then by Taylor's series, the value of $y$ at an neighbouring point $x+h$ is written as
	\begin{eqnarray}
		y(x_0+h) = y(x_0) + hy\sprime(x_0) + \frac{h^2}{2}y\dprime(x_0) + \frac{h^3}{3!}y\tprime(x_0) + \cdots
	\end{eqnarray}
	where $y\sprime(x)$ represent the derivative of $y$ with respect to $x$ since at $x_0$, $y_0$ is unknown, $y\sprime(x_0)$ is obtained by computing $f(x_0,y_0)$. Similarly higher derivatives of $y$ at $x_0$ can also be computed by using the relation.
	\begin{eqnarray}
		y\sprime &=& f(x,y)\sps
		y\dprime &=& f_x + f_y y\sprime\sps
		y\tprime &=& f_{xx} + 2f_{xy}y\sprime + f_{yy}(y\sprime)^2 + f_y y\dprime 
	\end{eqnarray}
	and so on. Then
	\begin{eqnarray}
		y(x_0+h) = y(x_0) + hf + \frac{h^2}{2!}(f_x + f_x y\sprime) + \frac{h^3}{3!}(f_{xx} + 2f_{xy}y\prime + f_{yy}(y\sprime)^2 + f_{yy}\dprime)
	\end{eqnarray}
	Hence, the value of $y$ at any neighbouring point $x_0+h$ can be obtained by summing the above infinite series.\\
	However, in any practical computation, the summation has to be terminated after some finite number of terms. If the series has been terminated after the $p^{th}$ derivative term then the approximated formula is called the Taylor Series approximation of $y$ of order $p$.
	
	\section{Euler's Method}
	Euler's method assumed our solution is written in the form of a Taylor's series. i.e,
	\begin{eqnarray}
		y(x+h) \approx y(x) + hy\sprime(x) + \frac{h^2}{2!}y\dprime(x) + \frac{h^3}{3!}y\tprime(x) + \cdots
	\end{eqnarray}
	This gives us a reasonably good approximation if we take plenty terms, and if the value of $h$ is reasonably small. \\
	
	\NI For Euler's method we take the first two terms only i.e $\dsp y(x+h) \approx y(x) + h\sprime(x)$\\
	The last term is just $h$ multiply $\dsp\frac{dy}{dx}$, so we can write Euler's method as follows
	\begin{eqnarray}
		y(x+h) \approx y(x) + hf(x,y)
	\end{eqnarray} 

	\NI In general, Euler's formula is given as 
	\begin{eqnarray}
		y_{i+1} = y_i + hf(x_i,y_i)
	\end{eqnarray}
	where $y_i$ is the current values, $y_{i+1}$ is the next estimated values, $h$ is the interval between steps and $f(x_i,y_i)$ is the value of the derivative at the current $(x_i,y_i)$ point.
	

	
	%%%%%%%%%%%%%%%%%%%CHAPTER THREE%%%%%%%%%%%%%%%%%%%
	\chapter{Methodology}
	\section{Introduction}
	The Adomian Decomposition Method consist of decomposing the unknown function $U(x,y)$ of any equation into a sum of an infinite number of component defined by the decomposition series
	\begin{eqnarray}
		U(x,y) = \sum_{n=0}^{\infty} U_n(x,y)
	\end{eqnarray}
	where the component $U_n(x,y), n\geq 0$ are to be determined in a recursive manner. The ADM is concern with finding the components $U_0, U_1, U_2,\ldots$ individually.\\
	
	\NI The ADM consist of splitting the given equation into linear and non-linear parts, inverting the highest order derivative operator contained in the linear operator on both sides, identifying the initial and/or boundary condition and the terms involving the independent variable alone as initial approximation, decomposing the unknown function into a series whose components are to be determined, decomposing the non-linear function in terms of special polynomials called Adomian polynomials and finding the successive terms of the series solution by recurrent relation using Adomian polynomials. The solution is found as an infinite series in which each term can be easily determined and that converges quickly towards an accurate (exact) solution.
	
	\section{Method of Solution of ADM}
	Consider a differential equation
	\begin{eqnarray}
		f(U(t)) = g(t)
	\end{eqnarray}
	where $f$ represents a general non-linear ordinary differential operator including both linear and non-linear terms. Thurs, the equation may be written as:
	\begin{eqnarray}
		LU + NU + RU = g
	\end{eqnarray}
	where $L$ is the linear operator, $N$ represent the non-linear operator and $R$ represent the remaining linear part. Solve for $LU$, we obtained
	\begin{eqnarray}
		LU = g - NU - RU
	\end{eqnarray}
	Then we defined the inverse operator of $L$ as $L^{-1}$ assuming it exist, we get
	\begin{eqnarray}
		L^{-1}LU =L^{-1}g - L^{-1}NU - L^{-1}Ru \label{eq:3_5}
	\end{eqnarray}
	
	\NI The Adomian Decomposition Method represents the $U(x,t)$ as an infinite series of the form
	\begin{eqnarray}
		U(x,t) = \sum_{n=0}^{\infty}U_n(x,t)\label{eq:3_6}
	\end{eqnarray}
	or equivalently,
	\begin{eqnarray}
		U(x,t) = U_0(x,t) + U_1(x,t) + U_2(x,t) + \cdots
	\end{eqnarray}
	Also, ADM defines the non-linear term $NU$ by the Adomian polynomials, which can be decomposed by an infinite series of polynomials given by
	\begin{eqnarray}
		NU = \sum_{n=0}^{\infty}A_n\label{eq:3_8}
	\end{eqnarray}
	where $A_n$ are the Adomian polynomials. Substituting equations \refx{3_6} and \refx{3_8} into equation \refx{3_5}, we get
	\begin{eqnarray}
		\sum_{n=0}^{\infty}U(x,t) = \phi_0 + L^{-1}g(x) + L^{-1} R\sum_{n=0}^{\infty}U_n - L^{-1}\sum_{n=0}^{\infty}A_n
	\end{eqnarray}
	\begin{eqnarray}
		\phi_0 = \left\{
			\begin{array}{l l}
				U(0) & \text{ if } L \frac{d}{dx}\sps
				%%%%
				U(0) + xU\sprime(0) & \text{ if } L \frac{d^2}{dx^2}\sps
				%%%%%%
				U(0) + xU\sprime(0) + \frac{x^2}{2!}U\dprime(0) & \text{ if } L \frac{d^3}{dx^3}\sps
				%%%%
				\vdots\sps
				%%%%%
				U(0) + xU\sprime(0) + \frac{x^2}{2!}U\dprime(0) + \cdots + \frac{x^n}{n!}U^n(0), & \text{ if } L \frac{d^{n+1}}{dx^{n+1}}
			\end{array}
		\right.
	\end{eqnarray}
	Therefore, the formal recurrence algorithm could be defined as
	\begin{eqnarray}
		\left\{
			\begin{array}{l}
				U_0 = \phi_0 + L^{-1} g(x),\sps
				U_1 = - L^{-1}RU_0 - L^{-1}A_0\sps
				U_2 = - L^{-1} RU_1 - L^{-1}A_1\\
				\vdots\\
				U_{n+1} = -L^{-1}RU_n - L^{-1}A_n,~~~ n\geq 0
			\end{array}
		\right.
	\end{eqnarray}
	where $A_n$ are Adomian polynomials generated for each non-linear term so that $A_0$ depends only on $U_0, A_1$ depends only on $U_0$ and $U_1, A_2$ depends only on $U_0,U_1$ and $U_2$ and etc.

	\section{Adomian Polynomials}
	The main part of ADM is calculating the Adomian Polynomials. The ADM decomposes the solution of $U$ and the non-linear terms $NU$ into series
	\begin{eqnarray}
		U = \sum_{n=0}^{\infty}U_n \hspace{1.2cm} N(U) = \sum_{n=0}^{\infty}A_n
	\end{eqnarray}
	where $A_n$ are the Adomian Polynomials.\\
	
	\NI To compute $A_n$, we take $NU=f(U)$ as a non-linear function in $U$ where $U=U(x)$ and we consider the Taylor Series expansion of $f(U)$ about the initial function $U_0$.
	\begin{multline}
		f(U) = f(U_0) + f\sprime(U_0)(U-U_0) + \frac{1}{2!}f\dprime(U_0)(U-U_0)^2 \sps
		+ \frac{1}{3!}f\tprime(U_0)(U-U_0)^3 + \cdots
	\end{multline}
	But $U=U_0+U_1+U_2+\cdots$. Then,
	\begin{multline}
		f(U) = f(U_0) + f\sprime(U_0)(U_1+U_2+U_3+\cdots) + \frac{1}{2!}f\dprime(U_0)(U_1+U_2+U_3+\cdots)^2 \sps
		+ \frac{1}{3!}f\tprime(U_0)(U_1+U_2+U_3+\cdots)^3 + \cdots
	\end{multline}
	by expanding all term we get
	\begin{multline}
		f(U) = f(U) + f\sprime(U_0)(U_1) + f\sprime(U_0)(U_2)+ f\sprime(U_0)(U_3) + \cdots \sps
		%%%%%%%
		+ \frac{1}{2!}f\dprime(U_0)(U_1)^2 + \frac{2}{2!}f\dprime(U_0)(U_1U_2) + \frac{1}{2!}f\dprime(U_0)(U_1U_3) + \cdots \sps
		%%%%%%%%
		+ \frac{1}{3!}f\tprime(U_0)(U_1)^3 + \frac{3}{3!}f\tprime(U_0)(U_1^2U_2) + \frac{1}{3!}f\tprime(U_0)(U_1^2U_3) + \cdots
	\end{multline}
	The Adomian Polynomials are constructed in a certain way so that the polynomial $A_1$ consists of all terms in the expansion of order 1, $A_2$ consist of all terms of order 2, and so on.\\
	
	\NI In general, $A_n$ consist of all terms of order $n$. Therefore, we have
	\begin{eqnarray}
		\begin{split}
			A_0 &= f(U_0)\sps
			A_1 &= U_1f\sprime(U_0)\sps
			A_2 &= U_2f\sprime(U_0)+ \frac{1}{2!}f\dprime(U_0)\sps
			%%%%%%%%%%
			A_3 &= U_3f\sprime(U_0) + \frac{2}{2!}U_1U_2f\dprime(U_0)+ \frac{1}{3!}U_1^3f\dprime(U_0)\sps
			%%%%%%%%%
			A_4 &= U_4f\sprime(U_0) + \left[\frac{1}{2!}U_2^2 + U_1U_3\right]f\dprime(U_0) + \frac{1}{2!}U_1^2U_2f\tprime(U_0) + \frac{1}{4!}U_1^4f''''(U_0)\sps
			\vdots
		\end{split}
	\end{eqnarray}
	Hence, $A_n$ was defined via the general formula
	\begin{eqnarray}
		A_n(U_0,U_1,\ldots,U_n) = \frac{1}{n!}\frac{d^n}{d\lambda^n}\left[  N\left( \sum_{n=0}^\infty U_n\lambda^n \right)  \right]_{\lambda=0} \hspace{.2cm} n=0,1,2,3,\ldots
	\end{eqnarray}
	To fin the $A_n$'s by Adomian general formula, these polynomial will be computed as follows:
	\begin{eqnarray*}
		\begin{split}
			A_0 &= N(U_0)\sps
			A_1 &= \frac{d}{d\lambda}N(U_0+U_1\lambda)\Big|_{\lambda=0} = N(U_0)U1\sps
			%%%%%%%%
			A_2 &= N\sprime(U_0)U_2 + \frac{1}{2!}N\dprime(U_0)U_1^2 - \frac{1}{2!}\frac{d^2}{dx^2}N(U_0+\lambda U_1 + \lambda^2 U_2)\Big|_{\lambda=0}\sps
			%%%%%%%%%%%%
			A_3 &= N\sprime(U_0)(U_3) + \frac{2}{2!}N\dprime(U_0)U_1U_2 + \frac{1}{3!}N\tprime(U_0)U_1^3\\  
			&= \frac{1}{3!}\frac{d^2}{dx^2}N(U_0+\lambda U_1 + \lambda^2U_2 + \lambda^3U_3)\Big|_{\lambda=0}\sps
			\vdots
		\end{split}
	\end{eqnarray*}
	
	\example{3.1}
	The Adomian polynomial of $f(U) = U^5$
	\begin{eqnarray*}
		f(U) &=&U^5\sps
		A_0 &=& U_0^5\sps
		A_1 &=& N\sprime(U_0)U_1 \implies 5U_0^4U_1\sps
		A_2 &=& N\sprime(U_0)U_2 + \frac{1}{2!}N\dprime(U_0)U_1^2  = 5U_0^4U_2+10U_0^3U_1^3\sps
		A_3 &=& N\sprime(U_0)U_3 + \frac{2}{2!}N\dprime(U_0)U_1U_2 + \frac{1}{3!}N\tprime(U_0)U_1^3 = 5U_0^4U_3 + 20U_0^3U_1U_2 \\
		&&+ 10U_0^3U_1^3\sps
		\vdots	
	\end{eqnarray*}




	
	
	%%%%%%%%%%%%%%%%%%%CHAPTER FOUR%%%%%%%%%%%%%%%%%%%
	\chapter{}
	
	




























	
	%%%%%%%%%%%%%%%%%%%CHAPTER FIVE%%%%%%%%%%%%%%%%%%%
	\chapter{Discussion of Results, Conclusion and Recommendation}
	\section{Discussion of Results}
	In table 1,2, and 3, we compare the solution obtained by Adomian Decomposition method and the exact solution. It is easy to see that the solution obtained from ADM are very close to the exact values in terms of the absolute errors.\\
	
	\NI The solution using Adomian decomposition method converges easily to the exact solution.
	
	\section{Conclusion}
	In this project, considering the error obtained from comparing the ADM and exact solution, the result obtained shows that the ADM and exact solution are in strong agreement with each other. This makes ADM a very powerful and efficient in finding solution for wide class of ordinary differential equation.
	
	\section{Recommendation}
	This project has undergo finding solution to second order ordinary differential equation of second derivatives using Adomian decomposition method. However, the method of Adomian decomposition can also be used to find solutions to ordinary differential equation of other greater than two.\\
	
	\NI I hereby recommend that further research on the use of Adomian decomposition method to find solution to ordinary differential equation of higher order.

	
	%%%%%%%%%%%%%%%%%%%REFERENCE%%%%%%%%%%%%%%%%%%%
	\chapter*{REFERENCES}
	\addcontentsline{toc}{chapter}{REFERENCES}
	
	\begin{description}
		\item Adomian G. (1986). \emph{Nonlinear stochatic operator equation}. Academic Press, San Diego.
		
		\item Hilder brandh F. B. (1974). \emph{Introduction to numerical Analysis}, Tata McGraw-Hill.
		
		\item Hossseini M. M. \& Nasabzadeh. H. (2007). \emph{Modified Adomian decomposition method for specific second order differential equation}. Applied mathematics and computation, 186(1),171-123.
		
		\item Inc E. L. (1956). \emph{Ordinary differences equations}, Dover publications, New York.
		
		\item King A. C, Billingham J., \& Otto S. R (2003), \emph{Differential Equation: linear, nonlinear,ordinary, partial}. Cambridge University Press, Cambridge.
		
		\item Mariam B. H. (2014), \emph{Iterative method for numerical solutions of boundary value problems}. Sharjah, United Arab Emirate.
		
		\item Oraphine, H., Tiza M. \& Amos P. (2016). \emph{Solving ordinary differential equation using power series}. Journal for innovative research in multidisciplinary field, Volume 2.
		
		\item Sasser E. J (2005). \emph{History of ordinary differential equations: The five hundred years}. Dover publications, Cincinnati.
		
		\item Simonons G. (1991). \emph{Differential equation with application and historical note}. (2nd ed.), McGraw-Hill, New York. 
		
		\item Vahidi A. B. \& Hasanzade M. (2012). \emph{Restared Adomian decomposition method for Bratu-Type problem}. Applied mathematical Science, 10, 479-486.
		
		\item Wazwaz A. M. (2011). \emph{Linear and Nonlinear integral equation. Method and application}. Springer, New York, USA.
		
		\item Zill D. \& Wright W. (2008). \emph{Differential equation and boundary value problems} (8th ed.). Brooks|Cole, Boston.
	\end{description}
	
\end{document}