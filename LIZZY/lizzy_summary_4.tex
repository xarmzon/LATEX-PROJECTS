\documentclass[12pt]{report}
\usepackage{amsmath}
\usepackage{amssymb}
\usepackage{graphicx}
\usepackage{longtable}
\usepackage{tikz}

\newcommand{\bt}[1]{\textbf{#1}}
\newcommand{\ubt}[1]{\textbf{\underline{#1}}}
\newcommand{\sps}{\\[0.2cm]}
\newcommand{\spn}[1]{\\[#1cm]}
\newcommand{\refn}[1]{(\ref{#1})}
\newcommand{\refx}[1]{\refn{eq:#1}}
\newcommand{\NI}{\noindent}
\newcommand{\dsp}{\displaystyle}
\newcommand{\sprime}{'}
\newcommand{\dprime}{''}
\newcommand{\tprime}{'''}
\newcommand{\tti}[1]{\textit{#1}}


\renewcommand*\contentsname{Table of Contents}
\renewcommand{\baselinestretch}{1.4}

\begin{document}

	\begin{titlepage}
		%%TITLE%%
		%\addcontentsline{toc}{chapter}{TITLE PAGE}
		\begin{center}
			\textbf{\itshape SUMMARY ON}
		\end{center} 
		\begin{center}
			{\bf \Large RADIATION EFFECTS ON UNSTEADY NATURAL CONVECTION FLOW IN AN INCLINED MEDIUM (2015)}
		\end{center}
		$$$$
		\vspace{3cm}
		\begin{center}
			\textbf{\itshape BY}
		\end{center} 
		$$$$
		\vspace{2cm}
		\begin{center}
			{\bf KODJO, Elizabeth\\
				17/56EB062}
		\end{center}
		$$$$
		\\ \\
	\end{titlepage}
	%\newpage
	%\pagenumbering{roman} 
	%%TABLE OF CONTENTS%%
	%\addcontentsline{toc}{chapter}{TABLE OF CONTENTS}
	%\tableofcontents
	
	\pagenumbering{arabic}
	%%%%%%%%%%%%%%%%%%%%%%%%%CHAPTER ONE%%%%%%%%%%%%%%%%%%%%%%%%%%%%
	\chapter{INTRODUCTION}
	In nature, there exist which are caused not only by the temperature, difference but also by concentrate on differences called heat and mass transfer.\sps
	The phenomenon of heat and mass transfer frequently exist in chemically processed industries such as food processing and polymer production.\sps
	magnetohydrodynamics has attracted so many attention due to its various application. It is used in engineering for food, it application in MHD PUMP and MHD bearing.
	
	\section{Objective of Study}
	The purpose of this study is introduce to solve the governing equation of radiation of effect on unsteady MHD natural convection flow in an inclined medium.
	
	\section{History of Fluid Mechanics}
	Hydroulics is derive from Hudour, a Greek word which means water. Hydraulics is that branch of engineering-science which deal with water either at rest or in motion.\sps
	Fluid is a substance which is capable of flowing. Fluid mechanics is a branch of engineering-Science that deals with the behaviour of fluid at rest and in motion. In solid, molecules are closed spaced while in liquid and gases the molecules spacing are large.\\
	
	\NI A solid can resist tensile, compressive and shear street up to a certain limit where as fluid has no tensile strength or little of it and it can resist the compressive force when kept only in a container liquid and gases exhibit different characteristics.\\
	
	\NI Fluid Mechanics can divided into three
	\begin{enumerate}
		\item \bt{Statics:} The study of incompressible fluid under static condition is called \bt{HYDROSTATICS} and study of compressible static gases is called \bt{AEROSTATICS}.
		
		\item \bt{Kinematics:} it deals with velocities, acceleration and their pattern of flow.
		
		\item \bt{Dynamics:} it deals with the relations between velocities, acceleration of fluid with force or energy causing them.
	\end{enumerate}
	
	\section{Definition of Terms}
	\begin{enumerate}
		\item \bt{Radiation:} The emission of transmission in form of waves through a material medium
		
		\item \bt{Magnetic hydrodynamcs(MHD):} The study of the magnetic properties of electrically conducting fluid
		
		\item \bt{Natural Convection:} A mechanism in which the fluid motion is not generated by any external source but only by density difference in the fluid due to temperature gradients.
		
		\item \bt{Inclined Medium:} A medium whose endpoints are at different heights.
	\end{enumerate}

	\section{Type of Fluids}
	Fluid can either be gaseous or liquid substance which is capable of flowing and conforming to the shape of container which may be classified into
	\begin{itemize}
		\item Liquid, gas vapour	
		\item Ideal and \& Real fluid
	\end{itemize}
	
	\begin{enumerate}
		\item Newtonian and Non-Newtonian Fluid: fluid that obey Newton's law of viscosity are Newtonian fluid. And fluid that doesn't are Non-Newtonian fluid.
		
		\item Compressible and Incompressible fluid: Fluids that changes in volume of a mass of fluid due to change in temperature or pressure are compressible while incompressible fluids are those whose element undergo no change in volume or density.
		
		\item Ideal or Real fluids: A fluid which has no resistance to shear stress are ideal fluid. Real fluid is one which has viscosity, surface tension and compressibility.
	\end{enumerate}

	\section{Properties of Fluid}
	\begin{enumerate}
		\item Density
		\item Specific Weight
		\item Specific Volume $\dsp \frac{\mathrm{volume}}{\mathrm{weight}}$
		\item Compressibility $PV = RT$
		\item Surface Tension
	\end{enumerate}

	\section{Types of Flow}
	\begin{enumerate}
		\item Laminar and Turbulent flow
		\item Uniform and non-uniform flow
		\item Steady and unsteady flow
		\item compressible and incompressible flow
	\end{enumerate}
	
	
	%%%%%%%%%%%%%%%%%%%%%%%%%CHAPTER TWO%%%%%%%%%%%%%%%%%%%%%%%%%%%%
	\chapter{LITERATURE REVIEW AND GOVERNING EQUATION OF FLUID MOTION}
	\section{Introduction}
	We consider three major equations governing fluid motion which are Navier Stoke Equation, Continuity Equation and The Energy Equation.\\
	
	\subsection{Continuity Equation}
	The continuity equation for an incompressible fluid is gotten from the law of conservation of mass $\dsp \frac{\partial \rho}{\partial t} + \rho \nabla \cdot \vec{v} = 0 $
	
	\subsection{Energy Equation}
	The energy equation is gotten from the first law of thermodynamics that state that the steady flow, the external workdone on any system plus the thermal energy transferred into or out of the system is equal to change of energy of the system. \quad\quad $\mathrm{work + heat} = \nabla\mathrm{energy}$
	
	\subsection{Navier-Stokes Equation}
	This equation was named after Calude-Louis Navier and George Gabriel Stokes in the Navier equation.\sps
	It is derived from the second law of motion which states that the rate at which mometum of the fluid mass is changing is equal to the net external force acting on it. Which can be written as
	\begin{eqnarray}
		\rho\left[\frac{\partial v_i}{\partial t_j} + v_j\frac{\partial v_j}{\partial x_k}\right] = \rho f_i -\frac{\partial\rho}{\partial x_j} + \mu \left[\frac{\partial^2v_i}{\partial x_i x_j} + \frac{\partial^2 v_j}{\partial x_jx_i}\right]\frac{\partial^2v}{\partial x^2_j}\notag
	\end{eqnarray}
	
	\section{Governing Equation of Fluid Motion}
	After derivation we get the following equation
	\begin{equation*}
		\frac{\partial u}{\partial t} = \frac{\partial^2 u}{\partial y^2} + \mathrm{Gr}\theta + \mathrm{Gmc} - \left(\mathrm{msm}\theta + \frac{1}{k}\right)u
	\end{equation*}
	\begin{equation*}
		\rho r\frac{\partial\theta}{\partial t} = \Big[1 + N\Big]\frac{\partial^2\theta}{\partial y^2}
	\end{equation*}
	\begin{equation*}
		Sc\frac{\partial c}{\partial t} = \frac{\partial^2 t}{\partial y^2}
	\end{equation*}

	%%%%%%%%%%%%%%%%%%%%%%%%%CHAPTER THREE%%%%%%%%%%%%%%%%%%%%%%%%%%%%
	\chapter{METHOD OF SOLUTION}
	\section{Final Solution Using Laplace Method}
	We consider the set of equation of the governing equation of fluid motion. After solving we get
	\begin{eqnarray}
		B(y\cdot sc,t) = \frac{Gr}{2M}\left\{2er + c \left(7 - e^{2_n\sqrt{M't}}\mbox{erfc}\left(n-\sqrt{M't}\right)\right) - e^{2_n\sqrt{M't}}\mbox{erfc}\left(n-\sqrt{M't}\right) \right\}\notag
	\end{eqnarray}
	
	
	%%%%%%%%%%%%%%%%%%%%%%%%%CHAPTER FOUR%%%%%%%%%%%%%%%%%%%%%%%%%%%%
	\chapter{Discussion of Result}
	In Non-dimensional form, the rate of heat transfer is given by
	\begin{equation*}
		N_u = -\left(\frac{\partial \theta}{\partial y}\right)y = 0 = \sqrt{\frac{\frac{\rho}{1 + N}}{\pi t}} = \sqrt{\frac{\rho r\pi t}{1 + N}}
	\end{equation*}
	\bt{Sherwood Number}
	This number is another important physical quantities which in Non-dimensional form is 
	\begin{equation*}
		S_n = -\left(\frac{\partial c}{\partial y}\right)y = \sqrt{\frac{sc}{\pi t}}
	\end{equation*}
	%%%%%%%%%%%%%%%%%%%%%%%%%CHAPTER FIVE%%%%%%%%%%%%%%%%%%%%%%%%%%%%
	\chapter{Summary}
	In this project work, we study the radiation effect on unsteady MHD natural convection fluid in an inclined medium. The governing equation are Non-dimensionalized and solve using Laplace. The results were presented in graphs and discussed for different value of parameter involved in the study.
	
	
	
\end{document}

